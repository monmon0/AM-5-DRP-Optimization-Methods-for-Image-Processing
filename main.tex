\documentclass[a4paper]{article}
\usepackage[affil-it]{authblk}
\usepackage{amsmath}
\begin{document}

\title{ AM-Reading-5: \\  Optimization Methods for Image Processing}

\author{Khoi Hoang %
  \thanks{UW email: \texttt{nk2hoang@uwaterloo.ca}} \and Ha Dang Vu\thanks{UW email: \texttt{hdvu@uwaterloo.ca}}   \and Monica Trinh \thanks{UW email: \texttt{m3trinh@uwaterloo.ca}} }
\affil{}

\author{Mentor: Phuong Dong Le%
  \thanks{UW email: \texttt{pdle@uwaterloo.ca}}}
\affil{Department of Mathematics, WiM Program, \\ University of Waterloo}

\date{Date: September - December, 2024}

\maketitle
\pagebreak
 % change margin, maybe font 2 small
\section{Week 1: Defining the minimization problem and image denoising, proximal gradient descent method.}

Answers to questions:
\begin{enumerate}
\item  
\subitem For Scheme 1:
We wish to write the regularization-term regression into a Least Square Problem and express them in terms of $\text{arg min} \{ \frac{1}{2} || u - x||^2_2 + \lambda || \Phi \vec{x}||^2_2$\}. \\ We first want to augment the data to incorporate the $L_2$ regularization penalty term into the matrix.  We  expressed $\tilde{X}$ as $\begin{bmatrix}
\vec{x} \\ \sqrt{\lambda}I \end{bmatrix}$ where I is the identity matrix and $\tilde{u} = \begin{bmatrix}
u \\ 0 \end{bmatrix}$ where $0 \in R^{1000}$. We then see that the new matrix  $\tilde{X} \in R^{(m + 1000) \text{ x } 1000}$. Consequentially, $\tilde{u} \in R^{m + 1000}$. Also note that $\Phi\vec{x}$ is the squared sum of the ?weight functions. \\

% boxing the function
\noindent\fbox{\begin{minipage}{\textwidth}
\paragraph{LS for Scheme 1}  We now obtain the standard Least Square Problem: \\ \begin{equation} \vec{x}^* = \text{arg min }\left\{ \frac{1}{2}||\tilde{X}\Phi \vec{x} - \tilde{u}||^2_2 \right\} \end{equation} \\. 
\end{minipage}} \\ 

\subitem For Scheme 2: We wish to write the regularization-term regression into a Least Square Problem and express them in terms of $\text{arg min} \{ \frac{1}{2} || u - x||^2_2 + \lambda || \Phi \vec{x}||_1 $\}. Since the penalty in this case is a L1 norm, we cannot directly linearized it into a least square problem. Instead, we will solve this problem by subjecting it to a constraint, i.e. $||\Phi\vec{x}||_1 \leq t $, and t is a parameter related to $\lambda $. The dimensions of the matrix remain the same. \\



% boxing the function
\noindent\fbox{\begin{minipage}{\textwidth}
\paragraph{LS for Scheme 2}     
We now obtain the standard Least Square Problem: 
\begin{center}
\\ \begin{equation} \vec{x}^* = \text{arg min }\left\{ \frac{1}{2}||u - \Phi \vec{x}||^2_2 \right\} \end{equation} subject to  $||\Phi\vec{x}||_1 \leq t $ as we wish to minimize the $L_1$ regularization term.
\end{center}
\end{minipage}} \\

*self-learning notes: By adding the L1 regularization term, LASSO regression can shrink the coefficients towards zero. When λ is sufficiently large, some coefficients are driven to exactly zero. 

This property of LASSO makes it useful for feature selection, as the variables with zero coefficients are effectively removed from the model. 
\item Q2:
\item Q3:
\item Q4:
\end{enumerate}

\end{document}